\section{Introduction}
This is the introduction.
\todo[inline]{write here}

%what is haptics
\subsection{Haptics}
The Oxford Dictionaries define haptics as 

\begin{displayquote}
"Relating to the sense of touch, in particular relating to the perception and manipulation of objects using the senses of touch and proprioception." \cite{Oxford}
\end{displayquote}
It dates the term back to the late 19th century	and states its origin from the greek words \textit{haptikos} meaning 'able to touch or grasp' and \textit{haptein}, 'fasten'.\\
In engineering, the research in the haptic field has started to emerge in late nineteen-eighties \cite{Srinivasan1995}. Before that, man-machine interaction was mainly limited to keyboard and mouse input, which is rather unidirectional and passive and it soon has become clear that this requires a more skilled user for all kinds of operations. This directly leads to limitations in performance. Haptics can extend this unilateral interaction by providing tactile or kinesthetic information. This combination can overcome the user's limitations and improve performances of high-precision tasks or high-force tasks drastically.\\
In \cite{Hayward2004} it is stated, that even though the research in the haptic domain has significantly increased in the past ten years, further investigations are necessary for the "quest for realism", especially in medical telesurgery applications where realism is key to performance. Other application domains include but are not limited to space operations, manufacturing, physical rehabilitation, arts or simply entertainment related devices.\\
The critical elements for stability of haptic setups, mainly focusing on haptic simulations are discussed in \cite{Adams1998}. It also motivates exploration of alternative control techniques due to the unpredictability of the human operator and the environment model.


%what are the problems in general in haptics
\subsection{Stability and Transparency Trade-Off}
In literature (\cite{Salcudean2000},  \cite{Christiansson2007}, \cite{Enayati2016})  the importance of stability and transparency is often emphasized, stating that they are the main problems related to haptics and haptic teleoperation applications. For conventional master slave operation systems Salcudean explains transparency as \begin{displayquote}
\textelp{} the master should feel to the operator as if the task were being manipulated directly.\cite{Salcudean2000}
\end{displayquote}

Even though the controller does not meet the definition of a conventional master-slave teleoperation setup, the problem settings of stability and transparency still apply.  Often, the stability issues arise from master-slave mass mismatch and stiffness of the environment. But high inertia also poses problems for higher frequencies. In this research however, the masses and frequencies are relatively low. \\
The stability of this controller is not only affected by the control scheme and the mechanical setup, but also by the operator, whose grasp can render a system stable or unstable \cite{Enayati2016}. Additionally, communication delays can also cause instability.\\
To gain insight into the stability, literature uses for example the root-locus method \cite{Christiansson2006} or the notion of passivity. For haptic devices, it can generally be assumed that the operator is passive \cite{Hogan1989}. Alternatively, one can also use the real data of the implementation to subjectively assess the stability. %In this work, the latter approach has been chosen.

%where has the previous research left off
\subsection{Project Description and Status Quo of Previous Projects}
Research on a handheld controller has been conducted at Yamamoto's lab in 2016 and 2017 (\cite{Asada2016}, \cite{Asada2016a}, \cite{Nakamura2016}). This controller is capable of giving a feedback to the user about the state of a remotely controlled robot. As stated in the papers, the feedback law is based on the roll and pitch angle of the robot. For the feedback two voice coil motors (VCMs) have been used. The disadvantage of these motors is that a constant voltage has to be applied to maintain a constant output force. This leads to an energy-inefficient system and produces a lot of extra heat. Furthermore, the motors were rated at $13.6$ V with $2$ N/A. The output force of each palm stimulator was exactly $2$ N. Due to the magnets in the VCMs the controller was rather heavy, which could fatigue the operator if used for too long.\\
The testing environment for this controller consisted in a Unity program that let the user control a tank with crawlers in an artificial landscape. When driving over the hills and smaller bumps, the user feels the magnitude of the tilt (orientation angles roll and pitch) as a pressure on the palms. \\
The goal of this work is to replace the VCMs by a set of series elastic actuators (SEAs). Different SEA-based controller designs shall be prototyped and tested on a real robot (topy). It is expected that the maximum output force is greater than in the current design. Furthermore, various feedback laws shall be implemented and tested to find the most intuitive feedback for the state of the robot. 

%what are the problems in this case
\subsection{Challenges Faced}
\todo[inline]{write about challenges expected and faced here}
%design had to be redone since VCMs have different implementation principles
%a lot of parameters to optimize
%no guideline on how to build SEA based systems for this purpose


\subsection{Goals, Requirements and constraints}
\todo[inline]{find a different name for this title}
%what is my goal, what are the requirements and constraints,

\subsection{Literature Review}
%literature review (similar projects, similar concepts, parallel work)
Research on haptic feedback in handheld devices has been conducted not only with force feedback (\cite{Prattichizzo2012}, \cite{Schoonmaker2006}), but also with pressure stimulation (\cite{Ajoudani2014}, \cite{Asada2016}, \cite{Nakamura2016}) or vibro-tactile feedback (\cite{Ajoudani2014}, \cite{Foottit2014}). The major applications for such haptic feedback devices can be found in (minimally invasive) surgical teleoperated medical systems (\cite{Bedem2009}, \cite{Enayati2016}, \cite{Tavakoli2004}, \cite{Nisky2011}), mobile augmented reality (\cite{Bermejo2017}, \cite{Hasegawa2006}) and teleoperated robotic systems (for example in space) \cite{Christiansson2007}. However this research project focuses on implementing a feedback on a handheld controller for a non-specified group of grounded wheeled robots. The controller uses pressurization feedback as a substitution for force feedback to give an intuitive feeling about the intrinsic state of the robot. In \cite{Mugge2016} intuitiveness is said to be achieved when the user immediately knows what the system's intentions are. The tests of the controller shall be made on the commercially available crawler robot developed by topy \footnote{\url{http://www.topy.co.jp}} using two tracks as means of transportation.
	
As an initial approach and to be consistent with the previous research in this area \cite{Asada2016a} the robots orientation (pitch and roll angles) have been taken as state of interest. As an alternative measure, the current in the two crawler motors can be fed back to the user, to have a rough idea of used energy. With feedback on the current consumption it is also detectable if the robot is stuck or being blocked somewhere.


\newpage