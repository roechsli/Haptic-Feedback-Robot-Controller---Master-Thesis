% Your name
\renewcommand{\YRname}{R. Oechslin}

% Your grade/post
\newcommand{\YRgrade}{M2}

% Submission date
\newcommand{\YRdate}{2018.Jul.31}

% Your research theme
\newcommand{\YRtheme}{Haptic Feedback Controller with Palm Pressurization}

% Work plan
\newcommand{\YRplan}{
	\hspace{-4truemm}
	\begin{tabularx}{170truemm}{|p{50truemm}||X|X|X|X|X|X|X|X|X|X|X|X|}
		\hline
		\multicolumn{13}{|c|}{\parbox[c][10truemm][c]{0truemm}{} \large Research theme: \bf \YRtheme} \\
		\hline
		\hline
		\multicolumn{13}{|c|}{\parbox[c][8truemm][c]{0truemm}{} \large \bf --- Research Plan ---} \\
		\hline
		Term \textbackslash Month & 2 & 3 & 4 & 5 & 6 & 7 & 8 & 9 & 10 & 11 & 12 & 1 \\
		\hline
		% For ``Work plan'', do not change above.
		\hline
		Literature review & & & & & & & & & & & & \\
		\shadecells{2-2}
		\hline
		Design PlayStation Controller  & & & & & & & & & & & & \\
		\shadecells{2-3}
		\hline
		Test PlayStation Controller & & & & & & & & & & & & \\
		\shadecells{4-6}
		\hline
		Frequency Response Analysis & & & & & & & & & & & & \\
		\shadecells{5-6}
		\hline
		Design Pilot Controller & & & & & & & & & & & & \\
		\shadecells{4-6}
		\hline
		Test Pilot Controller & & & & & & & & & & & & \\
		\shadecells{6-7}
		\hline
		& & & & & & & & & & & & \\
		%\shadecells{2-10}
		\hline
		Theoretical Analysis & & & & & & & & & & & & \\
		\shadecells{6-7}
		\hline
		Analyze data and compare & & & & & & & & & & & & \\
		\shadecells{7-8}
		\hline
		Write Thesis & & & & & & & & & & & & \\
		\shadecells{8-8}
		\hline
	\end{tabularx}
}

% Main contents of your work
\newcommand{\YRachievement}{
	
	\section{Introduction}
	This report is the continuation of the first two reports about the project "Haptic Feedback Controller with Palm Pressurization". The last report has left off with the idea of implementing a voltage follower and suggested to retake measurements to create a Bode diagram. Furthermore, it has been suggested to create an analytical model and analysis of the setup in order to run simulations to identify setup parameters such as the equivalent spring constant, gain values or motor parameters.
	

%	\begin{figure}[h!]
%		\centering
%		\includegraphics[width=0.6\linewidth]{Figs/mechanical_schematic}
%		\caption{Simplified mechanical schematic of the actuation system with the stimulator.}
%		\label{fig:mechanical_schematic}
%	\end{figure}


%	\begin{figure}[h!]
%		\centering
%		\begin{tabular}{|l|c|c|}%designator | explanation | unit
%			\hline
%			 Designator & Explanation & Unit \\ \hline \hline
%			$T_m$ & Motor torque & [Nm]\\ 
%			$T$ & Output torque acting on carriage& [Nm]\\
%			$b_{op}$ & Damping coefficient of the operator & [Ns/m]\\
%			$J_T$ & Total inertia of mechanical setup & [$\textit{kgm}^2$]\\
%			\hline
%		\end{tabular}
%		\caption{Setup parameters}
%		\label{tab:setup_params}
%	\end{figure}
	

\section{Discussion}	

\section{Conclusion}


\section{Outlook}



	
	%�����܂�
}


